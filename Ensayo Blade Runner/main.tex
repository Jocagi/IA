\documentclass[]{scrreprt}
\renewcommand*\thesection{\arabic{section}}

\usepackage[utf8]{inputenc}
\usepackage[a4paper, total={6in, 9.5in}]{geometry}

\begin{document}

\title{\textbf{Blade Runner 2049}}
\author{José Carlos Girón Márquez - 1064718}
\date{Enero 2021}
\maketitle

\section{Introducción}
Blade Runner 2049 es un película de ciencia ficción estadounidense y una secuela a la película original de 1982. Esta es una historia ubicada en un futuro distópico donde se ha logrado crear organismos sintéticos y nos cuestiona sobre lo que es la vida, la conciencia y qué es lo que nos hace verdaderamente humanos.

La película se ubica en al año 2049, donde existen humanos creados por bioingeniería, conocidos como replicantes, que actúan como esclavos en esta sociedad. La historia se centra en "K", un replicante, quien trabaja para la policía como un \textit{blade runner}, un oficial que se encarga de perseguir y retirar (cazar) a replicantes rebeldes y obsoletos. Después de retirar a un replicante, K encuentra una caja enterrada que contiene los restos de una replicante que murió durante una cesárea, lo que demuestra que los replicantes pueden reproducirse biológicamente, algo que se creía imposible. Se teme que esto pueda provocar conflictos, por lo que se le ordena a K que encuentre y retire al niño nacido de la replicante para ocultar la verdad. El resto de la película se centra en la búsqueda de este niño, y el desarrollo personal de K. Mientras él se acerca más a la verdad, más comienza a cuestionarse sobre sí mismo y sus recuerdos.

\section{Implicaciones}

\subsection{Implicaciones éticas}
Uno de los puntos centrales de la película es la humanidad de las replicantes, o la falta de ella. Ellos al ser producidos en masa por una corporación no tienen los mismos derechos que tendría un ser humano nacido naturalmente, y esa es precisamente la razón por la que los superiores de K desean eliminar toda evidencia de que los replicantes se pueden reproducir, ya que eso nos los haría diferentes a cualquier otro humano. Ellos fueron creados exclusivamente para obedecer órdenes, incluso se hace alusión a esto en una escena donde la superior de K le pregunta si se va a negar a retirar al niño nacido de la replicante, a lo que él responde que no sabía que esa era una posibilidad. El hecho de que a pesar de que son capaces en pensar por ellos mismos se les trate como esclavos abre la puerta a la idea de que su rebeldía esté justificada, con paralelismo a la lucha por los derechos de los esclavos en la vida real.

\subsection{Implicaciones legales}
Actualmente, en varios países europeos, el uso de técnicas para la manipulación genética en humanos está prohibida por el Convenio de Asturias, por otro lado también existe la Declaración Universal sobre el Genoma y Derechos Humanos de la ONU, el cual prohíbe la clonación de humanos y considera el genoma humano como un patrimonio de la humanidad. Muchos países tienen regulaciones estrictas sobre el uso aceptable de humanos en desarrollo científico, y aunque no todos los países tienen leyes que prohíban la creación de seres como los replicantes que aparecen en la película, sería muy difícil que fuese viable en la vida real. Y aún suponiendo que sea posible crear estos seres, en la mayoría de países se le otorga la ciudadanía a una persona nacida en territorio nacional por lo que se les trataría como personas. Y si bien se puede argumentar que los replicantes se les puede considerar propiedad privada, es muy seguro que muchos gobiernos opinarían lo contrario.

\subsection{Implicaciones tecnológicas}
Esta es una película de ciencia ficción que presenta tecnologías muy interesantes. Por mencionar algunas están los replicantes, seres creados por bioingeniería, Joi la inteligencia artificial que actúa como la pareja romántica de K, hologramas, los clásicos autos voladores, computadoras que utilizan datos biométricos como forma de autenticación, etc. Muchos de estos desarrollos serían herramientas muy útiles, y de hecho, muchas de las tecnologías presentadas ya existen de alguna forma o están en fase de investigación, por lo que no sería muy descabellado pensar que muchos de estos inventos existan en 30 años.

\section{Agente Inteligente}
Uno de los mejores ejemplos de un agente inteligente viene de uno de los personajes secundarios, Joi. Ella es la pareja romántica de K. Joi tiene una capacidad de razonamiento similar a la de un humano. También posee diversos sensores que le permiten procesar la información de su entorno, durante gran parte de la película se muestra que es capaz de ver y oír, e incluso tiene cierto tipo de tacto, evidenciado en la escena donde sale a la lluvia y se ve que se proyecta sobre sí misma la imagen de las gotas de agua. Sus acciones están dirigidas a cumplir su tarea como acompañante, y se limitan a mantener conversaciones y a la proyección holográfica de una figura femenina que la representa, por lo que es probable que el modelo de su política de control esté enfocado en maximizar la satisfacción de su usuario. 

Debido a que Joi interactúa en el mundo real, su entorno se define como parcialmente observable. Incluso, al inicio de la película, debido a las limitaciones de sus sensores, lo único que podía observar era la habitación de K.
Además su entorno se puede definir como benigno, ya que no existe nada que le impida interactuar con su amo cuando este está presente. La naturaleza del entorno en el cual interactúa Joi es estocástico, debido a que las entradas del sistema nunca so las mismas.

Crear una IA parecida a Joi no sería tarea sencilla, en este personaje se pueden observar diversas ramas de la IA como: 
\begin{itemize}
\item
\textbf{Razonamiento y Toma de Decisiones}, al actuar de forma autónoma y al ser capaz de ofrecer sus propias opiniones.
\item
\textbf{Representación del conocimiento}, por su habilidad de procesar información y usarla a su favor, o al ser capaz de proyectarla.
\item
\textbf{Robótica}, por los componentes físicos necesarios para que se mueva libremente.
\item
\textbf{Procesamiento de lenguajes}, por su capacidad de expresar sus ideas de forma verbal y poder entender lo que se le está diciendo.
\item
\textbf{Visión de computador}, porque puede identificar su entorno.
\end{itemize}

\section{Conclusión}
La inteligencia artificial es más que una idea presentada en películas de ciencia ficción, es una tecnología con la que interactuamos en el día a día. Todos los componentes de los agentes inteligentes de la película ya existen, solo es cuestión de tiempo para que se perfeccionen. Nuestro presente es más impresionante que lo proponían muchas películas del pasado. Puede ser que las verdaderas capacidades de IA aún no han sido imaginadas.

\end{document}














